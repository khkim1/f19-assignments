
\documentclass[11pt]{article}

\input{../../tex/defs.tex}

% Useful syntax commands
\newcommand{\jarr}[1]{\left[#1\right]}   % \jarr{x: y} = {x: y}
\newcommand{\jobj}[1]{\left\{#1\right\}} % \jobj{1, 2} = [1, 2]
\newcommand{\pgt}[1]{\, > {#1}}          % \pgt{1} = > 1
\newcommand{\plt}[1]{\, < {#1}}          % \plt{2} = < 2
\newcommand{\peq}[1]{\, = {#1}}          % \peq{3} = = 3
\newcommand{\prop}[1]{\langle{#1}\rangle}% \prop{x} = <x>
\newcommand{\matches}[2]{{#1}\sim{#2}}   % \matches{a}{b} = a ~ b
\newcommand{\aeps}{\varepsilon}          % \apes = epsilon
\newcommand{\akey}[2]{.{#1}\,{#2}}       % \akey{s}{a} = .s a
\newcommand{\aidx}[2]{[#1]\,{#2}}        % \aidx{i}{a} = [i] a
\newcommand{\apipe}[1]{\mid {#1}}        % \apipe{a} = | a

% Other useful syntax commands:
%
% \msf{x} = x (not italicised)
% \falset = false
% \truet = true
% \tnum = num
% \tbool = bool
% \tstr = str


\begin{document}

\hwtitle
  {Assignment 1}
  {Kun Ho Kim (khkim1)} %% REPLACE THIS WITH YOUR NAME/ID

\problem{Problem 1}

Part 1:

\begin{alignat*}{1}
\msf{Property}~p ::= \qamp \varepsilon \\
\mid \qamp > n \quad \mid \quad < n \\
\mid \qamp = n \quad \mid \quad =s \\
\mid \qamp p_{1} \land p_{2} \quad \mid \quad p_{1} \lor p_{2}
\\
\msf{Schema}~\tau ::= \qamp \tnum \prop{p} \mid \quad \tstr \prop{p} \quad \mid \quad \tbool \\
\mid \qamp \{(s: \tau_{s})^{*}\} \\
\mid \qamp [\tau]
\end{alignat*}

\begin{comment}
\mid \qamp [\tnum \prop{p}] \quad \mid \quad [\tstr \prop{p}] \quad \mid \quad [\tbool]\\
\mid \qamp [\{(s: \tau_{s})\}^{*}]
\end{comment}

Part 2:

% mathpar is the environment for writing inference rules. It takes care of
% the spacing and line breaks automatically. Use "\\" in the premises to
% space out multiple assumptions.
\begin{mathpar}

\ir{S-Bool-False}{\ }{\matches{\falset}{\tbool}}
\ir{S-Bool-True}{\ }{\matches{\truet}{\tbool}}
\\
\ir{N $\varepsilon$}{\ }{\matches{n}{\tnum \prop{\eps}}}
\ir{N>}{n > m}{\matches{n}{\tnum \prop{> m}}}
\ir{N<}{n < m}{\matches{n}{\tnum \prop{< m}}}
\ir{N=}{n = m}{\matches{n}{\tnum \prop{= m}}}
\\
\ir{N-and}{\matches{n}{\tnum \prop{p_{1}}} \land \matches{n}{\tnum \prop{p_{2}}}}{\matches{n}{\tnum \prop{p_{1} \land p_{2}}}}
\ir{N-or}{\matches{n}{\tnum \prop{p_{1}}} \lor \matches{n}{\tnum \prop{p_{2}}}}{\matches{n}{\tnum \prop{p_{1} \lor p_{2}}}}
\\
\ir{S $\varepsilon$}{\ }{\matches{s}{\tstr \prop{\eps}}}
\ir{S=}{s = w}{\matches{s}{\tstr \prop{= w}}}
\\
\ir{S-and}{\matches{s}{\tstr \prop{p_{1}}} \land \matches{s}{\tstr \prop{p_{2}}}}{\matches{s}{\tstr \prop{p_{1} \land p_{2}}}}
\ir{S-or}{\matches{s}{\tstr \prop{p_{1}}} \lor \matches{s}{\tstr \prop{p_{2}}}}{\matches{s}{\tstr \prop{p_{1} \lor p_{2}}}}
\\
\ir{Dict}{\forall s' \in s \quad \matches{j_{s'}}{\tau_{s'}}}{\matches{\{(s: j)^*\}}{\{(s: \tau_{s})^*\}}}
\ir{Array}{\forall i = 0, ..., |j|-1 \quad \matches{j_{i}}{\tau}}{\matches{[j^*]}{[\tau]}}


\end{mathpar}

\newpage

\problem{Problem 2}

Part 1:

\begin{mathpar}
\ir{Terminal}{\ }{\val{(\varepsilon, j)}} 

\ir{DictKey}{s' \in s}
{(.s'a, \{(s: j)^*\}) \mapsto (a, j_{s'})} 

\ir{ArrIdx}{ \ }
{([n]a, [j^*]) \mapsto (a, j_{n})} 

\ir{Pipe}{ (a, j_{n}) \mapsto (a', j_{n}') \quad \forall n = 0, ..., |j|-1}
{(|a, [j^*]) \mapsto (|a', [j'^*]))} 

\ir{Pipe-$\varepsilon$}{ }
{(|\varepsilon, [j^*]) \mapsto (\varepsilon, [j^*])} 

\end{mathpar}

Part 2:

\begin{mathpar}
\ir{Access$\varepsilon$}{ }{ \matches{\varepsilon}{ \tau }}
\ir{AccessArr}{ \matches{a}{\tau} }{ \matches{[n]a}{[\tau]}} 
\ir{AccessDict}{ s' \in s \quad \matches{a}{\tau_{s'}} }{ \matches{.s'a}{\{(s: \tau_{s})^*\}} } 
\ir{AccessMap}{ \matches{a}{\tau} }{ \matches{|a}{ [\tau] }}
\end{mathpar}

\textit{Accessor safety}: for all $a, j, \tau$, if $\matches{a}{\tau}$ and $\matches{j}{\tau}$, then there exists a $j'$ such that $\evals{(a, j)}{\aeps, j'}$.

\begin{proof}
We prove by induction separately for the four accessor types: 

$\bullet$ Base case: $\varepsilon$. This case trivially holds, since $(\varepsilon, j)$ denotes that the accessor has terminated. 

Assume for induction that for all $a, j, \tau$ such that $a$ is a composition of $k$ primitive accessors excluding "$|$", $a \sim \tau, j \sim \tau \Rightarrow \exists j'$ s.t $\evals{(a, j)}{(\varepsilon, j')}$. We exhaustively consider all successor triplets $(S(a), j, \tau)$ for which $S(a), j \sim \tau$ and $S(a)$ is a composition of $k+1$ primitive accessors excluding "$|$": 

$\bullet$ If $S(a) = [n]a \sim [\tau], [j^*] \sim [\tau]$ then by (ArrIdx) we have $([n]a, [j^*]) \mapsto (a, j_{n})$. Since $a \sim \tau$ by the inversion lemma on (AccessArr) and $j _{n} \sim \tau \quad \forall n = 0, ..., |j|-1$, we conclude that $\exists j'$ s.t $\evals{(a, j_{n})}{(\varepsilon, j')}$ by the induction hypothesis. 

$\bullet$ If $S(a) = .s'a \sim \{(s: \tau_{s})^*\}, \{(s: j)^*\} \sim \{(s: \tau_{s})^*\}$,  then by the inversion lemma on (AccessDict) we have $s' \in s$ and $a \sim \tau_{s'}$. By (DictKey) we then have $(.s'a, \{(s: j)^*\}) \mapsto (a, j_{s'})$. Since $a \sim \tau_{s'}, j _{s'} \sim \tau_{s'}$ we conclude that $\exists j'$ s.t $\evals{(a, j_{s'})}{(\varepsilon, j')}$ by the induction hypothesis. 

Now we provide a separate proof of the accessor safety theorem for Pipe. 

$\bullet$ If $|a \sim [\tau], [j^*] \sim [\tau]$, then $j_{n} \sim \tau ~~ \forall n = 0, ..., |j|-1$ and $a \sim \tau$ by the inversion lemma on (AccessMap). As shown previously for $S(a) = [n]a, S(a) = .s'a$, it holds that $j_{n} \sim \tau~~\forall n = 0, ..., |j|-1$ and $a \sim \tau \Rightarrow \exists (a', j_{n}')~~\forall n = 0, ..., |j|-1$ s.t $(a, j_{n}) \mapsto (a', j_{n}')$ and $\exists \tau'$ s.t $a', j_{n}' \sim \tau'$. Invoking inference rule (Pipe) we obtain $(|a, [j^*]) \mapsto (|a', [j^*'])$. Since $a', j_{n}' \sim \tau$ we may repeatedly apply (Pipe) until we obtain $\evals{(|a, [j^*])}{(|\varepsilon, [j_{t}^*])}$ since every accessor ends with $\varepsilon$. Then invoking (Pipe-$\varepsilon$), we obtain $\evals{(|a, [j^*])}{(\varepsilon, [j_{t}^*])}$. 

We've exhaustively covered all cases of $(a, j, \tau)$ for which $a, j \sim \tau$ and thus conclude the proof. 


\end{proof}

\end{document}


\begin{comment}
$\bullet$ Case 2: $[n]a$. First consider the base case when $a = [n]\varepsilon$. By (AccessArr), it follows that $\matches{[n]\varepsilon}{[\tau]} \quad \forall \tau$. Then, by (Arr-Idx), $j \sim [\tau] \Rightarrow ([n]\varepsilon, j) \mapsto (\varepsilon, j_{n})$. So the safety theorem holds.  

We now assume for induction that $[n]a \sim \tau, j \sim \tau \Rightarrow \exists j'$ s.t $\evals{([n]a, j)}{(\varepsilon, j')}$. We wish to show that $[n]aa' \sim \tau, j \sim \tau \Rightarrow \exists j'$ s.t $\evals{([n]aa', j)}{(\varepsilon, j')}$. We assume that $[n]a \sim [\tau]$ so by the inversion lemma we have $aa' \sim \tau$. For $j \sim [\tau]$, we further get by (Arr-Idx) that $([n]aa', j) \mapsto (aa', j_{n})$. Then we know that $aa' \sim \tau$ and $j_{n} \sim \tau$.  
\end{comment}
