\documentclass[11pt]{article}

\input{../../tex/defs.tex}

%%%% Useful syntax commands:

% \wci - \mathsf{instrs}
% \wcs - \mathsf{stack}
% \wcm - \mathsf{memory}
% \wcl - \mathsf{locals}

% For example, you can define a configuration as:
% \{\wci{:}~e^*; ~\wcs{:}~n_\wcs^*; ~\wcm{:}~n_\wcm^*; ~\wcl{:}~n_\wcl^*\}

% Each WebAssembly instruction, including the new ones, has a corresponding macro:
% \wconst{n}
% \wbinop{\oplus}
% \wblock{e^*}
% \wloop{e^*}
% \wbr{i}
% \wbrif{i}
% \wlabel{e_\msf{cont}^*}{(n^*; ~e_\msf{body}^*)}
% \wcall{i}
% \wif{e_\msf{then}^*}{e_\msf{else}^*}
% \wfor{e_\msf{init}^*}{e_\msf{cond}^*}{e_\msf{post}^*}{e_\msf{body}^*}
% \wtry{e_\msf{try}^*}{e_\msf{raise}^*}
% \wraise

\begin{document}

\hwtitle
  {Assignment 5}
  {Kun Ho Kim (khkim1)} %% REPLACE THIS WITH YOUR NAME/ID

% Here's some complete examples of Latexing with WebAssembly.
% Please delete this before submitting your homework.


\problem{Problem 2}

Part 1:

\begin{mathpar}

\ir{T-If}
  {\typeJC{e_{\msf{body}}^*}{\rho} ~~ \typeJC{e_{\msf{else}}^*}{\rho}}
  {\typeJC{\ifnt{e_\msf{body}^*}{e_\msf{else}^*}}{\tfun{\msf{i32}}{\rho}}}

\\
\ir{D-If-True}
  {n \neq 0}
  {\wsteps
    {\wcs{:}~n; ~\wci{:}~\ifnt{e_\msf{body}^*}{e_\msf{else}^*}}
    {\wci{:}~e_\msf{body}^*}
  }
\\
\ir{D-If-False}
  {n = 0}
  {\wsteps
    {\wcs{:}~n; ~\wci{:}~\ifnt{e_\msf{body}^*}{e_\msf{else}^*}}
    {\wci{:}~e_\msf{else}^*}
  }
\end{mathpar}

Part 2:

\begin{mathpar}
\ir{T-For}
  {\typeJC{e_{\msf{init}}^*}{\eps \rightarrow \eps} ~~ 
   \typeJC{e_{\msf{cond}}^*}{\eps \rightarrow \msf{i32}} ~~
   \typeJC{e_{\msf{post}}^*}{\eps \rightarrow \eps} ~~
   \typeJC{e_{\msf{body}}^*}{\eps \rightarrow \eps}
   }
  {\typeJC{\wfor{e_{\msf{init}}^*}{e_{\msf{cond}}^*}{e_{\msf{post}}^*}{e_{\msf{body}}^*}}{\eps \rightarrow \eps}}
\\

\ir{D-For}
  {\ }
  { \{\wci{:}~\wfor{e_{\msf{init}}^*}{e_{\msf{cond}}^*}{e_{\msf{post}}^*}{e_{\msf{body}}^*}\} \mapsto \\
  \{\wci{:}~\wblock{e_{\msf{init}}^*, (\wloop{(\wblock{e_{\msf{cond}}^*, (\wbrif{0}), (\wbr{2})}), e_{\msf{body}}^*, e_{\msf{post}}^*, (\wbr{0}))}}\}}
  

\end{mathpar}

\newpage
Part 3:

\begin{comment}
\ir{T-Try}
  {\typeJCM{\ctx~\with~\msf{labels}:~\ctx.\msf{labels}+1}{e_{\msf{try}}^*}{\eps \rightarrow \eps} ~~ 
   \typeJCM{\ctx~\with~\msf{labels}:~\ctx.\msf{labels}+1}{e_{\msf{raise}}^*}{\msf{i32} \rightarrow \eps}}
  {\typeJC{\wtry{e_{\msf{try}}^*}{e_{\msf{raise}}^*}}{\eps \rightarrow \eps}}
\end{comment} 

\begin{mathpar}
\ir{T-Try}
  {\typeJC{e_{\msf{try}}^*}{\eps \rightarrow \eps} ~~ 
   \typeJC{e_{\msf{raise}}^*}{\msf{i32} \rightarrow \eps}}
  {\typeJC{\wtry{e_{\msf{try}}^*}{e_{\msf{raise}}^*}}{\eps \rightarrow \eps}}
\\
\ir{T-Raise}
  {\ }
  {\typeJC{\wraise}{\msf{i32} \rightarrow \msf{i32}}}
\\
\ir{D-Try-Step}
  { \wsteps{C~\with~\msf{stack}:n^*;~\msf{instrs}:e_{\msf{try}}^*}
           {C'~\with~\msf{stack}:n'^*;~\msf{instrs}:e_{\msf{try}}'^*} }
  { \wsteps{C~\with~\wci{:}~\wtry{e_{\msf{try}}^*}{e_{\msf{raise}}^*};~\msf{stack}:n^*}
           {C'~\with~\wci{:}~\wtry{e_{\msf{try}}'^*}{e_{\msf{raise}}^*};~\msf{stack}:n^*'} 
  }
\\
\ir{D-Try-Done}
  { \ }
  { \wsteps{\wci{:}~\wtry{\eps}{e_{\msf{raise}}^*}}
           {\wci{:}~\eps} 
  }
\\
\ir{D-Raise-Try}
  { \ }
  { \wsteps{ \wci{:}~\wtry{ \wraise, \_ }{e_{\msf{raise}}^*} }
           { \wci{:}~e_{\msf{raise}}^* } 
  }
\\
\ir{D-Raise-Label}
  { }
  { \wsteps{\wci{:}~\wlabel{\_}{(n; \wraise, \_)}}
           {\msf{stack}:n;~\wci{:}~\wraise}
  }
\\
\ir{D-Raise-Frame}
  { }
  { \wsteps{\wci{:}~\wframe{\_}{\{\wci{:}~\wraise, \_;~\msf{stack}:n\}}} 
           {\msf{stack}:n;~\wci{:}~\wraise}
  }
\\
\ir{D-Raise-Val}
  {\ }
  { \val{\{\wci{:}~\wraise\}} }
\\
\ir{D-Return-Try}
  { \ }
  { \wsteps{ \wci{:}~\wtry{ \wreturn, \_ }{e_{\msf{raise}}^*} }
           { \wci{:}~\wreturn } 
  }
\\
\ir{D-Branch-Try}
  { \ }
  { \wsteps{ \wci{:}~\wtry{ (\wbr i), \_ }{e_{\msf{raise}}^*} }
           { \wci{:}~\wbr i } 
  }
\end{mathpar}

\newpage

\problem{Problem 3}

Part 1:

\begin{enumerate}
\item \textbf{Undefined behavior:} No. The initial memory can only affect the specific values pushed onto the stack during run time which do not influence the static compile time typecheck of $C$. Therefore, since $\typeJC{C}{\tau^*}$ was shown agnostic of the runtime stack values, it follows that $\typeJC{C'}{\tau^*}$ even if different stack values are observed at run time. 

\item \textbf{Private function call:} No. Assuming that $C.\msf{instrs}$ doesn't contain any administrative instructions, the private function can never be called in $C'$. This is because a function can only be called via the $\wcall{i}$ instruction and $i$ cannot be manipulated by the memory to be a different value since it is embedded in the instruction (and not taken from the value stack).  
\end{enumerate}

Part 2:

\begin{enumerate}
\item \textbf{Undefined behavior:} No. The reason is the same as the reason for Part 1. Typechecking occurs agonistic to the stack values. The function index passed to $\msf{call\_indirect}$ is the only aspect the memory can influence. Regardless of the function index popped off the stack to be fed into $\msf{call\_indirect}$, it will always step to either a trap instruction or a function call both of which are proper behaviors. 


\item \textbf{Private function call:} Yes. Consider the following sequence of instructions 
\begin{align*}
&(\msf{i32.const~2})\\
&(\msf{i32.load})\\
&(\msf{call\_indirect~(param~i32)~(result~i32)})
\end{align*}
Let's say the original memory was initialized with all 1s and the $\msf{i32.load}$ was intended to load a $1$ so function 1 would've been called. An adversary could manipulate the memory to be all 0s such that $0$ is loaded from the $\msf{i32.load}$ instruction onto the stack. Then as long as $\wcall{0}$ has the type $\msf{i32} \rightarrow \msf{i32}$, it follows that $\msf{call\_indirect~i32} \rightarrow \msf{i32}$ will step to $\msf{call~0}$. The key issue with $\msf{call\_indirect}$ is that the function index $i$ is no longer embedded into the instruction, but taken from the stack whose values can be manipulated by changing the memory. 
\end{enumerate}


\end{document}
