\documentclass[11pt]{article}

%%% PACKAGES

\usepackage{fullpage}  % 1 page margins
\usepackage{latexsym}  % extra symbols, e.g. \leadsto
\usepackage{verbatim}  % verbatim mode
\usepackage{proof}     % proof mode
\usepackage{amsthm,amssymb,amsmath}  % various math symbols
\usepackage{color}     % color control
\usepackage{etoolbox}  % misc utilities
\usepackage{graphics}  % images
\usepackage{mathpartir}% inference rules
\usepackage{hyperref}  % hyperlinks
\usepackage{titlesec}  % title/section controls
\usepackage{minted}    % code blocks
\usepackage{tocstyle}  % table of contents styling
\usepackage[hang,flushmargin]{footmisc} % No indent footnotes
\usepackage{parskip}   % no parindent
\usepackage{tikz}      % drawing figures
\usepackage[T1]{fontenc}
%%% COMMANDS

% Typography and symbols
\newcommand{\msf}[1]{\mathsf{#1}}
\newcommand{\ctx}{\Gamma}
\newcommand{\qamp}{&\quad}
\newcommand{\qqamp}{&&\quad}
\newcommand{\Coloneqq}{::=}
\newcommand{\proves}{\vdash}
\newcommand{\str}[1]{{#1}^{*}}
\newcommand{\eps}{\varepsilon}
\newcommand{\brc}[1]{\{{#1}\}}
\newcommand{\binopm}[2]{{#1}~\bar{\oplus}~{#2}}
\newcommand{\aequiv}{\equiv_\alpha}

% Untyped lambda calculus
\newcommand{\fun}[2]{\lambda ~ {#1} ~ . ~ {#2}}
\newcommand{\app}[2]{{#1} ~ {#2}}
\newcommand{\fix}[1]{\msf{fix}~{#1}}

% Typed lambda calculus - expressions
\newcommand{\funt}[3]{\lambda ~ ({#1} : {#2}) ~ . ~ {#3}}
\newcommand{\ift}[3]{\msf{if} ~ {#1} ~ \msf{then} ~ {#2} ~ \msf{else} ~ {#3}}
\newcommand{\ifnt}[2]{\msf{if} ~ {#1} ~ \msf{else} ~ {#2}}
\newcommand{\rec}[5]{\msf{rec}(#1; ~ #2.#3.#4)(#5)}
\newcommand{\lett}[4]{\msf{let} ~ \hasType{#1}{#2} = {#3} ~ \msf{in} ~ {#4}}
\newcommand{\falset}{\msf{false}}
\newcommand{\truet}{\msf{true}}
\newcommand{\case}[5]{\msf{case} ~ {#1} ~ \{ L({#2}) \to {#3} \mid R({#4}) \to {#5} \}}
\newcommand{\pair}[2]{({#1},{#2})}
\newcommand{\proj}[2]{{#1} . {#2}}
\newcommand{\inj}[3]{\msf{inj} ~ {#1} = {#2} ~ \msf{as} ~ {#3}}
\newcommand{\letv}[3]{\msf{let} ~ {#1} = {#2} ~ \msf{in} ~ {#3}}
\newcommand{\fold}[2]{\msf{fold}~{#1}~\msf{as}~{#2}}
\newcommand{\unfold}[1]{\msf{unfold}~{#1}}
\newcommand{\poly}[2]{\Lambda~{#1}~.~{#2}}
\newcommand{\polyapp}[2]{{#1}~[{#2}]}

% Typed lambda calculus - types
\newcommand{\tnum}{\msf{number}}
\newcommand{\tstr}{\msf{string}}
\newcommand{\tint}{\msf{int}}
\newcommand{\tbool}{\msf{bool}}
\newcommand{\tfun}[2]{{#1} \rightarrow {#2}}
\newcommand{\tprod}[2]{{#1} \times {#2}}
\newcommand{\tsum}[2]{{#1} + {#2}}
\newcommand{\trec}[2]{\mu~{#1}~.~{#2}}
\newcommand{\tvoid}{\msf{void}}
\newcommand{\tunit}{\msf{unit}}
\newcommand{\tpoly}[2]{\forall~{#1}~.~{#2}}

% WebAssembly
\newcommand{\wconst}[1]{\msf{i32.const}~{#1}}
\newcommand{\wbinop}[1]{\msf{i32}.{#1}}
\newcommand{\wgetlocal}[1]{\msf{get\_local}~{#1}}
\newcommand{\wsetlocal}[1]{\msf{set\_local}~{#1}}
\newcommand{\wgetglobal}[1]{\msf{get\_global}~{#1}}
\newcommand{\wsetglobal}[1]{\msf{set\_global}~{#1}}
\newcommand{\wload}{\msf{i32.load}}
\newcommand{\wstore}{\msf{i32.store}}
\newcommand{\wsize}{\msf{memory.size}}
\newcommand{\wgrow}{\msf{memory.grow}}
\newcommand{\wunreachable}{\msf{unreachable}}
\newcommand{\wblock}[1]{\msf{block}~{#1}}
\newcommand{\wloop}[1]{\msf{loop}~{#1}}
\newcommand{\wbr}[1]{\msf{br}~{#1}}
\newcommand{\wbrif}[1]{\msf{br\_if}~{#1}}
\newcommand{\wreturn}{\msf{return}}
\newcommand{\wcall}[1]{\msf{call}~{#1}}
\newcommand{\wlabel}[2]{\msf{label}~\{#1\}~{#2}}
\newcommand{\wframe}[2]{\msf{frame}~({#1}, {#2})}
\newcommand{\wtrapping}{\msf{trapping}}
\newcommand{\wbreaking}[1]{\msf{breaking}~{#1}}
\newcommand{\wreturning}[1]{\msf{returning}~{#1}}
\newcommand{\wconfig}[5]{\{\msf{module}{:}~{#1};~\msf{mem}{:}~{#2};~\msf{locals}{:}~{#3};~\msf{stack}{:}~{#4};~\msf{instrs}{:}~{#5}\}}
\newcommand{\wfunc}[4]{\{\msf{params}{:}~{#1};~\msf{locals}{:}~{#2};~\msf{return}~{#3};~\msf{body}{:}~{#4}\}}
\newcommand{\wmodule}[1]{\{\msf{funcs}{:}~{#1}\}}
\newcommand{\wcg}{\msf{globals}}
\newcommand{\wcf}{\msf{funcs}}
\newcommand{\wci}{\msf{instrs}}
\newcommand{\wcs}{\msf{stack}}
\newcommand{\wcl}{\msf{locals}}
\newcommand{\wcm}{\msf{mem}}
\newcommand{\wcmod}{\msf{module}}
\newcommand{\wsteps}[2]{\steps{\brc{#1}}{\brc{#2}}}
\newcommand{\with}{\underline{\msf{with}}}
\newcommand{\wvalid}[2]{{#1} \vdash {#2}~\msf{valid}}
\newcommand{\wif}[2]{\msf{if}~{#1}~{\msf{else}}~{#2}}
\newcommand{\wfor}[4]{\msf{for}~(\msf{init}~{#1})~(\msf{cond}~{#2})~(\msf{post}~{#3})~{#4}}
% assign4.3 custom
\newcommand{\wtry}[2]{\msf{try}~{#1}~\msf{catch}~{#2}}
\newcommand{\wraise}{\msf{raise}}
\newcommand{\wraising}[1]{\msf{raising}~{#1}}

% Inference rules
%\newcommand{\inferrule}[3][]{\cfrac{#2}{#3}\;{#1}}
\newcommand{\ir}[3]{\inferrule*[right=\text{(#1)}]{#2}{#3}}
\newcommand{\s}{\hspace{1em}}
\newcommand{\nl}{\\[2em]}
\newcommand{\steps}[2]{#1 \boldsymbol{\mapsto} #2}
\newcommand{\evals}[2]{#1 \boldsymbol{\overset{*}{\mapsto}} #2}
\newcommand{\subst}[3]{[#1 \rightarrow #2] ~ #3}
\newcommand{\dynJ}[2]{#1 \proves #2}
\newcommand{\dynJC}[1]{\dynJ{\ctx}{#1}}
\newcommand{\typeJ}[3]{#1 \proves \hasType{#2}{#3}}
\newcommand{\typeJC}[2]{\typeJ{\ctx}{#1}{#2}}
\newcommand{\typeJCM}[3]{\typeJ{\{#1\}}{#2}{#3}}
\newcommand{\hasType}[2]{#1 : #2}
\newcommand{\val}[1]{#1~\msf{val}}
\newcommand{\num}[1]{\msf{Int}(#1)}
\newcommand{\err}[1]{#1~\msf{err}}
\newcommand{\trans}[2]{#1 \leadsto #2}
\newcommand{\size}[1]{|#1|}

\newcommand{\hwtitle}[2]{\begin{center}{\Large #1} \\[0.5em] {\large #2}\end{center}\vspace{1em}}
\newcommand{\toc}{{\hypersetup{hidelinks}\tableofcontents}}
\newcommand{\problem}[1]{\section*{#1}}

%%% CONFIG

% Spacing around title/sections
\titlelabel{\thetitle.\quad}
\titlespacing*{\section}{0pt}{10pt}{0pt}
\titlespacing*{\subsection}{0pt}{10pt}{0pt}


% Show color on hyperlinks
\hypersetup{colorlinks=true}

% Stylize code blocks
\usemintedstyle{xcode}
% TODO(wcrichto): remove line of spacing beneath code blocks
\newcommand{\nm}[2]{
  \newminted{#1}{#2}
  \newmint{#1}{#2}
  \newmintinline{#1}{#2}}
\nm{lua}{}
\nm{ocaml}{}
\nm{rust}{}
\nm{prolog}{}

\newcommand{\ml}[1]{\ocamlinline|#1|}

\usetocstyle{standard}


\begin{document}

\hwtitle
  {Assignment 2}
  {Kun Ho Kim (khkim1)} 

\problem{Problem 2}

\textbf{Part 1:}

\begin{mathpar}

\text{Step 1:}\qquad
\ir{D-App-Body}
  {\ir{D-App-Lam}
    {\ir{D-App-Done}
      {\ir{D-Lam}{ \ }{\val{\fun{\_}{x}}}}
      {\dynJ{\{x \rightarrow D\}}{\steps
        {\app{(\fun{x}{\fun{\_}{x}})}{L}}
        {\fun{\_}{x}}}}}
    {\dynJ{\{x \rightarrow D\}}{\steps
      {\app{\app{(\fun{x}{\fun{\_}{x}})}{L}}{*}}
      {\app{(\fun{\_}{x})}{*}}}}}
  {\dynJ{\varnothing}{\steps
    {\app{(\fun{x}{\app{\app{(\fun{x}{\fun{\_}{x}})}{L}}{*}})}{D}}
    {\app{(\fun{x}{\app{(\fun{\_}{x})}{*}})}{D}}}}
    
\text{Step 2:}\qquad
\ir{D-App-Body}
    {\ir{D-App-Body}
        {\ir{D-Var}
            { x \rightarrow D \in \{x \rightarrow D\} }
            {\dynJ{\{x \rightarrow D\}}{x \mapsto D}}}
        {\dynJ{\{x \rightarrow D\}}{\steps
            {\app{(\fun{\_}{x})}{*}}
            {\app{(\fun{\_}{D})}{*}}}
        }
    }
    {\dynJ{\varnothing}{\steps
            {\app{(\fun{x}{\app{(\fun{\_}{x})}{*}})}{D}}
            {\app{(\fun{x}{\app{(\fun{\_}{D})}{*}})}{D}}}}
    
\text{Step 3:}\qquad
\ir{D-App-Body}
    {\ir{D-App-Done}
        {\val{D}}
        {\dynJ{\varnothing}{\steps
            {\app{(\fun{\_}{D})}{*}}
            {D}}
        }
    }
    {\dynJ{\varnothing}{\steps
            {\app{(\fun{x}{\app{(\fun{\_}{D})}{*}})}{D}}
            {\app{(\fun{x}{D})}{D}}}}
            
\text{Step 4:}\qquad
\ir{D-App-Done}
    {\val{D}}
    {\dynJ{\varnothing}{\steps
            {\app{(\fun{x}{D})}{D}}
            {D}}}
\end{mathpar}



\textbf{Part 2:}

\begin{mathpar}
\ir{D-Let-Done}
  {\val{e_{\msf{body}}}}
  {\letv{x}{e_{\msf{var}}}{e_{\msf{body}}} \mapsto e_{\msf{body}}}
  
\ir{D-Let-Body}
  {\dynJ{\Gamma, x \rightarrow e_{\msf{var}}}{e_{\msf{body}} \mapsto e_{\msf{body}}'}}
  {\dynJ{\Gamma}{\letv{x}{e_{\msf{var}}}{e_{\msf{body}}} \mapsto \letv{x}{e_{\msf{var}}}{e_{\msf{body}}'}}}
\end{mathpar}

\newpage

\problem{Problem 3}

\textbf{Claim 1.} The $\msf{let}$ extension violates preservation. 

\begin{proof}
We prove the claim by presenting a counter example. Consider the following expression: 
\begin{mathpar}
e = (\letv{x : \tnum}{(\fun{y}{y})}{x})
\end{mathpar}
Assume for contradiction that preservation holds for $e$. Since $\dynJ{\hasType{x}{\tnum}}{\hasType{x}{\tnum}}$, we have $\hasType{e}{\tnum}$ by (T-Let). Thus by preservation, if $e \mapsto e'$, then $\hasType{e'}{\tnum}$. Applying (D-Let) on $e$ we obtain 
\begin{mathpar}
\evals{e}{\fun{y}{y}} 
\end{mathpar}
By preservation $e' = \fun{y}{y}$ has type $\tnum$. However a function cannot have type $\tnum$ since from the typing rules you can only derive a type of the form $\tau_{\msf{arg}} \rightarrow \tau_{\msf{ret}}$ for functions. Thus preservation does not hold for $e$ which concludes the proof. 
\end{proof}

\textbf{Claim 2.} The type safety theorem holds for the $\msf{rec}$ extension. 


\\
%%%%%%%%%% PRESERVATION PROOF %%%%%%%%%%%%
$\bullet$ Preservation: $\hasType{\rec{e_\msf{base}}{x_\msf{num}}{x_\msf{acc}}{e_\msf{acc}}{e_\msf{arg}}}{\tau} \land \rec{e_\msf{base}}{x_\msf{num}}{x_\msf{acc}}{e_\msf{acc}}{e_\msf{arg}} \mapsto e' \Rightarrow \hasType{e'}{\tau}$

By inversion on (T-Rec), 
\begin{align*}
&\hasType{e_\msf{arg}}{\tnum}\\\
&\hasType{e_\msf{base}}{\tau}\\ 
&\typeJ{\hasType{x_\msf{num}}{\tnum},\hasType{x_\msf{acc}}{\tau}}{e_\msf{acc}}{\tau}
\end{align*}

Assume for induction that Preservation holds for $e_\msf{base}, e_\msf{acc}, e_\msf{arg}$. We then enumerate all possible cases in which the $\msf{rec}$ expression can step. 

A. Assume $e_{\msf{arg}} \mapsto e_{\msf{arg}}'$, so $\rec{e_\msf{base}}{x_\msf{num}}{x_\msf{acc}}{e_\msf{acc}}{e_\msf{arg}} \mapsto \rec{e_\msf{base}}{x_\msf{num}}{x_\msf{acc}}{e_\msf{acc}}{e_\msf{arg}'}$ by (D-Rec-Step). By IH, $\hasType{e_{\msf{arg}}'}{\tnum}$. By (T-Rec), $\hasType{\rec{e_\msf{base}}{x_\msf{num}}{x_\msf{acc}}{e_\msf{acc}}{e_\msf{arg}'}}{\tau}$. 

B. Assume $e_{\msf{arg}} = 0$ so $\rec{e_\msf{base}}{x_\msf{num}}{x_\msf{acc}}{e_\msf{acc}}{e_\msf{arg}} \mapsto e_{\msf{base}}$ by (D-Rec-Base). $\hasType{e_{\msf{base}}}{\tau}$. 

C. Assume $e_{\msf{arg}} = n > 0$, so, 
\begin{align*}
\rec{e_\msf{base}}{x_\msf{num}}{x_\msf{acc}}{e_\msf{acc}}{n} \mapsto [x_\msf{num} \rightarrow n, x_\msf{acc} \rightarrow \rec{e_\msf{base}}{x_\msf{num}}{x_\msf{acc}}{e_\msf{acc}}{n-1}] \ e_\msf{acc}
\end{align*}
Since $\typeJ{\hasType{x_\msf{num}}{\tnum},\hasType{x_\msf{acc}}{\tau}}{e_\msf{acc}}{\tau}$ by inversion, we have 
\begin{align*}
\hasType{[x_\msf{num} \rightarrow n, x_\msf{acc} \rightarrow \rec{e_\msf{base}}{x_\msf{num}}{x_\msf{acc}}{e_\msf{acc}}{n-1}] \ e_\msf{acc}}{\tau}
\end{align*}
by the substitution typing lemma. 

%%%%%%%%%% PROGRESS PROOF %%%%%%%%%%%%
$\bullet$ Progress: $\hasType{\rec{e_\msf{base}}{x_\msf{num}}{x_\msf{acc}}{e_\msf{acc}}{e_\msf{arg}}}{\tau} \Rightarrow (\exists e'$ s.t $e \mapsto e') \lor \val{e}$

By inversion on (T-Rec), 
\begin{align*}
&\hasType{e_\msf{arg}}{\tnum}\\\
&\hasType{e_\msf{base}}{\tau}\\ 
&\typeJ{\hasType{x_\msf{num}}{\tnum},\hasType{x_\msf{acc}}{\tau}}{e_\msf{acc}}{\tau}
\end{align*}

Assume for induction that Preservation holds for $e_\msf{base}, e_\msf{acc}, e_\msf{arg}$. We enumerate all possible cases in which the the $\msf{rec}$ expression can step. 

A. $e_{\msf{arg}} \mapsto e_{\msf{arg}}' \Rightarrow \rec{e_\msf{base}}{x_\msf{num}}{x_\msf{acc}}{e_\msf{acc}}{e_\msf{arg}} \mapsto \rec{e_\msf{base}}{x_\msf{num}}{x_\msf{acc}}{e_\msf{acc}}{e_\msf{arg}'}$ by (D-Rec-Step). 

B. $\val{e_{\msf{arg}}} \land (e_{\msf{arg}} = n = 0) \Rightarrow \rec{e_\msf{base}}{x_\msf{num}}{x_\msf{acc}}{e_\msf{acc}}{e_\msf{arg}} \mapsto e_{\msf{base}}$ by (D-Rec-Base).

C. $\val{e_{\msf{arg}}} \land (e_{\msf{arg}} = n > 0) \Rightarrow \rec{e_\msf{base}}{x_\msf{num}}{x_\msf{acc}}{e_\msf{acc}}{n} \mapsto [x_\msf{num} \rightarrow n, x_\msf{acc} \rightarrow \rec{e_\msf{base}}{x_\msf{num}}{x_\msf{acc}}{e_\msf{acc}}{n-1}] \ e_\msf{acc}$ by (D-Rec-Dec). 








\begin{comment}
\begin{mathpar}

\ir{T-Rec}
  {\typeJC{e_\msf{arg}}{\tnum} \\ \typeJC{e_\msf{base}}{\tau} \\
   \typeJ{\ctx,\hasType{x_\msf{num}}{\tnum},\hasType{x_\msf{acc}}{\tau}}{e_\msf{acc}}{\tau}}
  {\typeJC{\rec{e_\msf{base}}{x_\msf{num}}{x_\msf{acc}}{e_\msf{acc}}{e_\msf{arg}}}{\tau}}

\ir{D-Rec-Step}
  {\steps{e}{e'}}
  {\steps
    {\rec{e_\msf{base}}{x_\msf{num}}{x_\msf{acc}}{e_\msf{acc}}{e}}
    {\rec{e_\msf{base}}{x_\msf{num}}{x_\msf{acc}}{e_\msf{acc}}{e'}}}

\ir{D-Rec-Base}
  {\ }
  {\steps
    {\rec{e_\msf{base}}{x_\msf{num}}{x_\msf{acc}}{e_\msf{acc}}{0}}
    {e_\msf{base}}}

\ir{D-Rec-Dec}
  {n >0}
  {\steps
    {\rec{e_\msf{base}}{x_\msf{num}}{x_\msf{acc}}{e_\msf{acc}}{n}}
    {[x_\msf{num} \rightarrow n, x_\msf{acc} \rightarrow \rec{e_\msf{base}}{x_\msf{num}}{x_\msf{acc}}{e_\msf{acc}}{n-1}] \ e_\msf{acc}}}
\end{mathpar}
\end{comment}

\end{document}
