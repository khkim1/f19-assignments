\documentclass[11pt]{article}

\input{../../tex/defs.tex}

\begin{document}

\hwtitle
  {Assignment 2}
  {Kun Ho Kim (khkim1)} 

\problem{Problem 2}

\textbf{Part 1:}

\begin{mathpar}

\text{Step 1:}\qquad
\ir{D-App-Body}
  {\ir{D-App-Lam}
    {\ir{D-App-Done}
      {\ir{D-Lam}{ \ }{\val{\fun{\_}{x}}}}
      {\dynJ{\{x \rightarrow D\}}{\steps
        {\app{(\fun{x}{\fun{\_}{x}})}{L}}
        {\fun{\_}{x}}}}}
    {\dynJ{\{x \rightarrow D\}}{\steps
      {\app{\app{(\fun{x}{\fun{\_}{x}})}{L}}{*}}
      {\app{(\fun{\_}{x})}{*}}}}}
  {\dynJ{\varnothing}{\steps
    {\app{(\fun{x}{\app{\app{(\fun{x}{\fun{\_}{x}})}{L}}{*}})}{D}}
    {\app{(\fun{x}{\app{(\fun{\_}{x})}{*}})}{D}}}}
    
\text{Step 2:}\qquad
\ir{D-App-Body}
    {\ir{D-App-Body}
        {\ir{D-Var}
            { x \rightarrow D \in \{x \rightarrow D\} }
            {\dynJ{\{x \rightarrow D\}}{x \mapsto D}}}
        {\dynJ{\{x \rightarrow D\}}{\steps
            {\app{(\fun{\_}{x})}{*}}
            {\app{(\fun{\_}{D})}{*}}}
        }
    }
    {\dynJ{\varnothing}{\steps
            {\app{(\fun{x}{\app{(\fun{\_}{x})}{*}})}{D}}
            {\app{(\fun{x}{\app{(\fun{\_}{D})}{*}})}{D}}}}
    
\text{Step 3:}\qquad
\ir{D-App-Body}
    {\ir{D-App-Done}
        {\val{D}}
        {\dynJ{\varnothing}{\steps
            {\app{(\fun{\_}{D})}{*}}
            {D}}
        }
    }
    {\dynJ{\varnothing}{\steps
            {\app{(\fun{x}{\app{(\fun{\_}{D})}{*}})}{D}}
            {\app{(\fun{x}{D})}{D}}}}
            
\text{Step 4:}\qquad
\ir{D-App-Done}
    {\val{D}}
    {\dynJ{\varnothing}{\steps
            {\app{(\fun{x}{D})}{D}}
            {D}}}
\end{mathpar}



\textbf{Part 2:}

\begin{mathpar}
\ir{D-Let-Done}
  {\val{e_{\msf{body}}}}
  {\letv{x}{e_{\msf{var}}}{e_{\msf{body}}} \mapsto e_{\msf{body}}}
  
\ir{D-Let-Body}
  {\dynJ{\Gamma, x \rightarrow e_{\msf{var}}}{e_{\msf{body}} \mapsto e_{\msf{body}}'}}
  {\dynJ{\Gamma}{\letv{x}{e_{\msf{var}}}{e_{\msf{body}}} \mapsto \letv{x}{e_{\msf{var}}}{e_{\msf{body}}'}}}
\end{mathpar}

\newpage

\problem{Problem 3}

\textbf{Claim 1.} The $\msf{let}$ extension violates preservation. 

\begin{proof}
We prove the claim by presenting a counter example. Consider the following expression: 
\begin{mathpar}
e = (\letv{x : \tnum}{(\fun{y}{y})}{x})
\end{mathpar}
Assume for contradiction that preservation holds for $e$. Since $\dynJ{\hasType{x}{\tnum}}{\hasType{x}{\tnum}}$, we have $\hasType{e}{\tnum}$ by (T-Let). Thus by preservation, if $e \mapsto e'$, then $\hasType{e'}{\tnum}$. Applying (D-Let) on $e$ we obtain 
\begin{mathpar}
\evals{e}{\fun{y}{y}} 
\end{mathpar}
By preservation $e' = \fun{y}{y}$ has type $\tnum$. However a function cannot have type $\tnum$ since from the typing rules you can only derive a type of the form $\tau_{\msf{arg}} \rightarrow \tau_{\msf{ret}}$ for functions. Thus preservation does not hold for $e$ which concludes the proof. 
\end{proof}

\textbf{Claim 2.} The type safety theorem holds for the $\msf{rec}$ extension. 


\\
%%%%%%%%%% PRESERVATION PROOF %%%%%%%%%%%%
$\bullet$ Preservation: $\hasType{\rec{e_\msf{base}}{x_\msf{num}}{x_\msf{acc}}{e_\msf{acc}}{e_\msf{arg}}}{\tau} \land \rec{e_\msf{base}}{x_\msf{num}}{x_\msf{acc}}{e_\msf{acc}}{e_\msf{arg}} \mapsto e' \Rightarrow \hasType{e'}{\tau}$

By inversion on (T-Rec), 
\begin{align*}
&\hasType{e_\msf{arg}}{\tnum}\\\
&\hasType{e_\msf{base}}{\tau}\\ 
&\typeJ{\hasType{x_\msf{num}}{\tnum},\hasType{x_\msf{acc}}{\tau}}{e_\msf{acc}}{\tau}
\end{align*}

Assume for induction that Preservation holds for $e_\msf{base}, e_\msf{acc}, e_\msf{arg}$. We then enumerate all possible cases in which the $\msf{rec}$ expression can step. 

A. Assume $e_{\msf{arg}} \mapsto e_{\msf{arg}}'$, so $\rec{e_\msf{base}}{x_\msf{num}}{x_\msf{acc}}{e_\msf{acc}}{e_\msf{arg}} \mapsto \rec{e_\msf{base}}{x_\msf{num}}{x_\msf{acc}}{e_\msf{acc}}{e_\msf{arg}'}$ by (D-Rec-Step). By IH, $\hasType{e_{\msf{arg}}'}{\tnum}$. By (T-Rec), $\hasType{\rec{e_\msf{base}}{x_\msf{num}}{x_\msf{acc}}{e_\msf{acc}}{e_\msf{arg}'}}{\tau}$. 

B. Assume $e_{\msf{arg}} = 0$ so $\rec{e_\msf{base}}{x_\msf{num}}{x_\msf{acc}}{e_\msf{acc}}{e_\msf{arg}} \mapsto e_{\msf{base}}$ by (D-Rec-Base). $\hasType{e_{\msf{base}}}{\tau}$. 

C. Assume $e_{\msf{arg}} = n > 0$, so, 
\begin{align*}
\rec{e_\msf{base}}{x_\msf{num}}{x_\msf{acc}}{e_\msf{acc}}{n} \mapsto [x_\msf{num} \rightarrow n, x_\msf{acc} \rightarrow \rec{e_\msf{base}}{x_\msf{num}}{x_\msf{acc}}{e_\msf{acc}}{n-1}] \ e_\msf{acc}
\end{align*}
Since $\typeJ{\hasType{x_\msf{num}}{\tnum},\hasType{x_\msf{acc}}{\tau}}{e_\msf{acc}}{\tau}$ by inversion, we have 
\begin{align*}
\hasType{[x_\msf{num} \rightarrow n, x_\msf{acc} \rightarrow \rec{e_\msf{base}}{x_\msf{num}}{x_\msf{acc}}{e_\msf{acc}}{n-1}] \ e_\msf{acc}}{\tau}
\end{align*}
by the substitution typing lemma. 

%%%%%%%%%% PROGRESS PROOF %%%%%%%%%%%%
$\bullet$ Progress: $\hasType{\rec{e_\msf{base}}{x_\msf{num}}{x_\msf{acc}}{e_\msf{acc}}{e_\msf{arg}}}{\tau} \Rightarrow (\exists e'$ s.t $e \mapsto e') \lor \val{e}$

By inversion on (T-Rec), 
\begin{align*}
&\hasType{e_\msf{arg}}{\tnum}\\\
&\hasType{e_\msf{base}}{\tau}\\ 
&\typeJ{\hasType{x_\msf{num}}{\tnum},\hasType{x_\msf{acc}}{\tau}}{e_\msf{acc}}{\tau}
\end{align*}

Assume for induction that Preservation holds for $e_\msf{base}, e_\msf{acc}, e_\msf{arg}$. We enumerate all possible cases in which the the $\msf{rec}$ expression can step. 

A. $e_{\msf{arg}} \mapsto e_{\msf{arg}}' \Rightarrow \rec{e_\msf{base}}{x_\msf{num}}{x_\msf{acc}}{e_\msf{acc}}{e_\msf{arg}} \mapsto \rec{e_\msf{base}}{x_\msf{num}}{x_\msf{acc}}{e_\msf{acc}}{e_\msf{arg}'}$ by (D-Rec-Step). 

B. $\val{e_{\msf{arg}}} \land (e_{\msf{arg}} = n = 0) \Rightarrow \rec{e_\msf{base}}{x_\msf{num}}{x_\msf{acc}}{e_\msf{acc}}{e_\msf{arg}} \mapsto e_{\msf{base}}$ by (D-Rec-Base).

C. $\val{e_{\msf{arg}}} \land (e_{\msf{arg}} = n > 0) \Rightarrow \rec{e_\msf{base}}{x_\msf{num}}{x_\msf{acc}}{e_\msf{acc}}{n} \mapsto [x_\msf{num} \rightarrow n, x_\msf{acc} \rightarrow \rec{e_\msf{base}}{x_\msf{num}}{x_\msf{acc}}{e_\msf{acc}}{n-1}] \ e_\msf{acc}$ by (D-Rec-Dec). 








\begin{comment}
\begin{mathpar}

\ir{T-Rec}
  {\typeJC{e_\msf{arg}}{\tnum} \\ \typeJC{e_\msf{base}}{\tau} \\
   \typeJ{\ctx,\hasType{x_\msf{num}}{\tnum},\hasType{x_\msf{acc}}{\tau}}{e_\msf{acc}}{\tau}}
  {\typeJC{\rec{e_\msf{base}}{x_\msf{num}}{x_\msf{acc}}{e_\msf{acc}}{e_\msf{arg}}}{\tau}}

\ir{D-Rec-Step}
  {\steps{e}{e'}}
  {\steps
    {\rec{e_\msf{base}}{x_\msf{num}}{x_\msf{acc}}{e_\msf{acc}}{e}}
    {\rec{e_\msf{base}}{x_\msf{num}}{x_\msf{acc}}{e_\msf{acc}}{e'}}}

\ir{D-Rec-Base}
  {\ }
  {\steps
    {\rec{e_\msf{base}}{x_\msf{num}}{x_\msf{acc}}{e_\msf{acc}}{0}}
    {e_\msf{base}}}

\ir{D-Rec-Dec}
  {n >0}
  {\steps
    {\rec{e_\msf{base}}{x_\msf{num}}{x_\msf{acc}}{e_\msf{acc}}{n}}
    {[x_\msf{num} \rightarrow n, x_\msf{acc} \rightarrow \rec{e_\msf{base}}{x_\msf{num}}{x_\msf{acc}}{e_\msf{acc}}{n-1}] \ e_\msf{acc}}}
\end{mathpar}
\end{comment}

\end{document}
